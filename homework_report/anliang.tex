\documentclass[a4paper, 10pt]{article}
\usepackage[top=0.5in, bottom=0.5in, left=1in, right=1in]{geometry}
% \usepackage[utf8]{inputenc}
% \usepackage[slantfont,boldfont]{xeCJK}
\usepackage{xeCJK}

\usepackage{fontspec}
% \setCJKmainfont{Noto Sans CJK SC}
% \setCJKmainfont{Noto Sans CJK SC Regular}
\setCJKmainfont[BoldFont=SimHei, ItalicFont=KaiTi]{SimSun}
% \setCJKmainfont{AR PL UMing CN}
\setmainfont{Times New Roman}
% \setmainfont{Arial}
% \setsansfont{Arial}
\usepackage{latexsym}
\usepackage{amsmath,amssymb,amsfonts}
\usepackage{color,xcolor}
\usepackage{graphicx}
\usepackage{tabu}
\usepackage{booktabs}
\usepackage{bm}
\usepackage{footnote}
\usepackage{url}
% \usepackage{subcaptionbox}
% \usepackage{subfigure}

\title{2D人体骨架识别}
\author{
安亮\\
2017310876
\and
秦钰超(coauthor)\\
2017210955
}
\begin{document}
\maketitle
\pagestyle{plain}
\setlength{\parindent}{2em}
\makeatletter
\renewcommand{\section}{\@startsection{section}{1}{0mm}
{0.5\baselineskip}{0.5\baselineskip}
{\Large\bf\leftline}}
\makeatother

\section{摘要}
本工作参考VNect实现了一个用于2D人体骨架检测的巻积神经网络。在测试数据上,该网络能够对部分人体关节点实现比较准确的检测。代码参见github\footnote{\url{https://github.com/anl13/Pose3D}}。

\section{研究问题}
人体骨架,又称人体关键点,指的是能反应人体姿态的关节点,例如肘、肩等。人体关键点检测(或人体骨架识别)指的是在一张或者多张图片
中,利用计算机算法标注出人体关键点在图像中的位置。这是计算机视觉领域广受关注的任务,也是实现人机交互的核心问题。
随着神经网络技术的广泛应用,利用神经网络解决诸如人体关键点检测等问题成为研究热点。CPM \cite{wei2016convolutional}首先实现了利用
巻积神经网络检测单人人体关键点,随后openpose\cite{cao2016realtime}成功使用巻积神经网络实现了单张图片多人体关节点检测。
最近,利用单张图片实现3D人体关键点检测也获得了很大关注,\cite{newell2016stacked}设计了特殊的巻积网络结构——stacked hourglass实现了较为
精确的单人体3D骨架识别。VNECT\cite{mehta2017vnect}则将2D人体关键点检测与坐标估计结合起来,实现了实时单人3D关键点检测与跟踪。

本工作研究的问题是单图片单人2D人体骨架检测问题。本工作参考VNECT中的2D骨架识别网络的网络结构,利用tensorflow\cite{abadi2016tensorflow}
予以实现,并在少量数据上进行了效果测试。

\section{采用数据}
本文主要采用的数据是mpi-inf-3dhp\cite{mehta2017monocular}。这是一个单人体数据集,带有2D和3D的关节标注。由于时间问题,本文并没有
利用该数据进行训练,而是直接迁移了VNECT训练出的模型参数,并在该数据集的个别图像上进行了测试。


\section{研究方法}
\subsection{网络结构}
本文使用了
图\ref{pic:net_struct}展示了我们使用的网络结构。其中,我们最终利用的输出仅为H层,也就是2D关节点heatmap层。在这个网络中,针对每个关节点会输出一个heatmap,
heatmap上能量最高的位置就是该关节点出现概率最大的位置。因此,最终输出的heatmap张量的维度为(N,H,W,C),N是输入图像的数量(在测试的时候一般为1),H是和heatmap图像的高度
(在该网络中是输入图像的$1/8$,W相同),W是heatmap图像的高度,C是heatmap图像的通道数,在该网络中为21,对应待检测的关节点数量。

这个网络有两个优点。第一,网络的前端使用了ResNet\cite{he2016deep}作为特征提取器,通过迁移ResNet在imagenet上训练好的参数可以加速网络训练的收敛速度。
第二,使用了全巻积的结构(除了计算关节长度部分),使得网络适应任何大小的图片输入。

\begin{figure}[ht!]
    \centering
    \includegraphics[width=\linewidth]{pic/net_struct.png}
    \label{pic:net_struct}
    \caption{网络结构
    % \footnote{图片来源与论文\cite{mehta2017vnect}}
    。}
\end{figure}

对于该网络,我们考虑直接把参数从VNECT网络训练出的参数迁移过来。然而,这里面也面临一些问题。原来的VNECT网络是用caffe架构训练出来的,参数表达的结构
也是caffe的结构。两种网络的参数在卷积核的维度顺序等方面有不少区别,我们在转换的时候进行了很多次尝试。然而,最终我们转移过来的参数并没有表现出完美的
结果,但仍然在部分关节的检测中展示出了不错的结果\footnote{这部分工作主要由秦钰超同学完成}。

\subsection{数据处理}
为了使图片转换为神经网络容易处理的形势,我们对图像进行了预处理和后处理\footnote{这部分工作主要由我完成}。

在预处理方面,考虑到我们的神经网络输出的heatmap的宽和高是输入图像的$1/8$,为了对输出图像做更好的处理,对于一个输入的图像$I$,我们首先会对图像进行尺度上的放缩,
将宽和高放缩到8的整数倍。其次,原输入图像每个像素点的值的范围是$(0,255)$,这种过大的取值范围对于神经网络是不友好的。因此,我们将像素点的值放缩到$(-0.4,0.6)$。

在后处理方面,我们将heatmap的尺度和值范围还原到原图,并提取heatmap最大值的坐标作为关节估计位置。

\section{分析结果}


\section{讨论}



\bibliographystyle{splncs}
\bibliography{ref.bib}

\end{document}
